\section{Background}

\subsection{Subtopic}
\lipsum[1]

\noindent If you want to add an image, you can do it like this:

\begin{figure}[H]
          \centering
          \includegraphics[width=1\linewidth]{img/bioeng.jpg}
          \caption{Image 1}
          \label{fig:1}
    \end{figure}
        
\noindent Figures can be referenced like this: The Department of Bioengineering Logo shows a human and prosthetic hand (Figure \ref{fig:1}).

\noindent If you need two figures next to each other: (see next page, the figures will fit in the next available space!)
\begin{figure*}
\begin{multicols}{2}
    \centering
        \begin{subfigure}[b]{0.475\textwidth}
            \centering
            \includegraphics[width=\textwidth]{img/bioeng.jpg}
            \caption{Part 1}    
        \end{subfigure}
        \hfill
        \begin{subfigure}[b]{0.475\textwidth}  
            \centering 
            \includegraphics[width=\textwidth]{img/bioeng.jpg}
            \caption{Part 2}    
        \end{subfigure}
\end{multicols}
    \caption{Images 1 and 2}
    \label{fig:2}
\end{figure*}

\subsection{Subtopic}
\lipsum[1]
\subsubsection{Subtopic}

\noindent Let's look at a table in latex:

\begin{table}[H]
    \centering
    \begin{tabular}{c|c c|c c}
         &\multicolumn{4}{c}{Random Thing 2} \\\hline
         Random Thing 1 & Std & Mean & Std & Mean  \\\hline
         R1&10&5&9&5\\
         R2&10&5&9&5\\
         R3&10&5&9&5\\
    \end{tabular}
    \caption{\label{tab:1}A random table}
\end{table}

\noindent Tables can be referenced like this: one study showed that x and y are correlated (Table \ref{tab:1}).